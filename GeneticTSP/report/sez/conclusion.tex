\section{Conclusion}
The project gave me the opportunity to apply the theory seen during the course to a real world problem like Genetic Algorithms. The initial design of the parallel architecture was strictly correlated to the results obtained from the analysis of the sequential program. Also at the beginning it was challenging to deal with the parallel implementation of the algorithm with the standard C++ threads, the error generated by the parallel program were hard to find (and debugging) manually, but thanks to tools like \textit{Valgrind} or \textit{ThreadSanitizer} they were quite easy to found and fix. However, these fact enabled me to understand the purpose of parallel skeleton framework like Fastflow that it hides all the parallelism complexity to the programmer by using object oriented programming technique. 

\begin{thebibliography}{9}
	
	\bibitem{fastflow} 
	Marco Aldinucci, Marco Danelutto, Peter Kilpatrick, Massimo Torquati.
	\textit{Fastflow: high-level and efficient streaming on multi-core }. Programming Multi-core and Many-core Computing Systems
	\\\texttt{https://github.com/fastflow/fastflow}
	
	\bibitem{genetic-algorithm-tutorial} 
	Darrell Whitley.
	\textit{A genetic algorithm tutorial }. Kluwer Academic Publishers
	
	\bibitem{algo-eng} 
	Müller-Hannemann, Matthias, Schirra, Stefan.
	\textit{Algorithm Engineering}. Springer
	
	% \bibitem{PaperL-BFQS} 
	% Dong C. Liu and Jorge Nocedal.
	% \textit{On the limited memory BFGS method for large scale optimization}. Mathematical Programming 45 (1989), pp. 503-528.
	% \\\texttt{https://people.sc.fsu.edu/\textasciitilde inavon/5420a/liu89limited.pdf}
	
\end{thebibliography} 